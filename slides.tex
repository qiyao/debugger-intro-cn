\documentclass[cjk]{beamer}  % 使用Beamer包
%\hypersetup{pdfpagemode=FullScreen}
\usepackage{CJK}   % 中文环境
\usepackage{indentfirst}
%\usepackage{beamerthemesplit}

\usepackage{makeidx}
%\usepackage[dvipdfm,CJKbookmarks,bookmarks=true,bookmarksnumbered=true]{hyperref}
%\usepackage[CJKbookmarks,bookmarks=true,bookmarksnumbered=true]{hyperref}
\usepackage{listings}


\hypersetup{colorlinks, linkcolor=blue, citecolor=blue,
    urlcolor=blue,
    plainpages=flase,
    pdfcreator=tex,
    bookmarksopen=true,
    pdfhighlight=/P,
    pdfauthor={Yao Qi <qiyaoltc@gmail.com>},
    pdfcreator={cTeX},
%    pdftitle={iptables 入门},
%    pdfkeywords={iptables 入门 Rae szlug},
    pdfstartview=FitH,
    pdfpagemode=UseOutlines,%UseOutlines, %None, FullScreen, UseThumbs
}

%\usepackage[all,bottom]{draftcopy}
%\draftcopyName{Copyright by Rae <crquan@gmail.com>, 2006}{50}
%\draftcopySetGrey{0.8} \draftcopyPageTransform{55 rotate}
%\draftcopyPageX{80}\draftcopyPageY{-25}

\lstset{
  basicstyle=\tiny,          % print whole listing small
  showstringspaces=false,
  numbers=left,
  numberstyle=\tiny,
  % labelstep=1,
  % commentstyle=\textsl,
  keywordstyle=\color{green}\bfseries\underbar,     % underlined bold red keywords
  identifierstyle={},         % nothing happens to other identifiers
  commentstyle=\mdseries\color{blue}, % white comments
  stringstyle=\color{red}\ttfamily}      % typewriter type for strings
% stringspaces=false}         % no special string spaces



%\usetheme{Madrid}  % 采用的主题
\usetheme{Warsaw}
%\usetheme{Rochester}
%\usetheme{Berkeley}
%\usetheme{PaloAlto}
%\usetheme{Montpellier}
%\usetheme{Berlin}
%\usetheme{Ilmenau}
%\usetheme{Dresden}

%\usecolortheme{albatross} % 采用的配色。
%\usecolortheme{beaver} white background
%\usecolortheme{beetle}
%\usecolortheme{dolphin}
%\usecolortheme{fly}
%\usecolortheme{lily}

%\useoutertheme{tree}

\usefonttheme{structureitalicserif}

\begin{document}
\begin{CJK}{UTF8}{gbsn}

\title{调试器}
\author{\href{mailto:qiyaoltc@gmail.com}{齐尧}}
\date{HelloGCC 2010 \\ \today}

% 生成上面定义的"\title", "\author", "\date"等信息
\frame{\titlepage}


% 生成目录菜单
\section{目录及流程}
\frame{
    \tableofcontents
}

\section{关于我和我的话题}

\begin{frame}
  \frametitle{关于我和我的话题}
  \begin{columns}
    \column{5cm}
    \begin{overprint}

      \begin{block}{我}
        \begin{itemize}
        \item 齐尧 \href{mailto:qiyaoltc@gmail.com}{qiyaoltc@gmail.com}
        \item 曾经在\href{http://sourceware.org/frysk/}{frysk}项目为PowerPC64维护断点和底层事件循环
        \item 中间中断了三年的toolchain开发,现在又开始继续toolchain的工作
        \item 现在还算调试器的初学者
        \end{itemize}
      \end{block}

    \end{overprint}


    \column{5cm}
    \begin{overprint}

      \begin{block}{我的话题}
        \begin{itemize}
        \item 它\textbf{不可能}让你在三十分钟内,让你明白调试器的内部,
        \item 甚至\textbf{不可能}让你明白调式器的一个组件或者模块,
        \item 它\textbf{能}让你从最简单的角度,理解调试器的必要功能,
        \item 掌握学习调试器的方法
        \end{itemize}
      \end{block}

    \end{overprint}
  \end{columns}
\end{frame}

\section{ 最简单的调试器 }

\subsection{最简单的调试器是什么样子的}
\frame{
  \frametitle{调试器}
  \begin{itemize}
  \item 父进程(调试器) 与 子进程(被调试程序)
  \item 通过操作系统提供的接口,
  \end{itemize}

  \begin{table}
    \begin{tabular}{|c|c|l|} \hline
      功能&操作&实现\\ \hline
      控制被调试程序&&ptrace, signal\\ \hline
      监视被调试程序&&breakpoint\\ \hline
      &&\\ \hline
    \end{tabular}
  \end{table}
}

\frame[containsverbatim]{
  \frametitle{最简单的调试器是什么样子的}

  \lstinputlisting[language=C,caption={simple-debugger.c},
  firstline=10,lastline=36,
  captionpos={b},label={listing:introduction:simple-debugger}]
  {../../book/introduction/simple-debugger.c}

  \begin{itemize}
  \item ab
  \item cd
  \end{itemize}
}
\subsection{调试器依赖的一些OS功能,比如ptrace}
\frame{
  \frametitle{}
  \begin{itemize}
  \item ptrace 
  \item Emacs + cscope;
  \end{itemize}
}
    
\subsection{和标准调试器的比较}
\frame{
  \frametitle{和标准调试器的比较}
  \begin{itemize}
  \item 断点
  \item ELF
  \item Frame Stack unwind
  \item Shared Object;
  \item Thread;
  \end{itemize}
}

\section{细节}

\subsection{断点}
\frame{
  \frametitle{断点到底是什么?}

  \begin{overprint}

    \begin{block}{断点}
      \begin{itemize}
      \item 与指令集密切相关的特殊指令或者非法指令
      \item 当这个指令被执行的时候,导致操作系统 trap,调试器就知道程序执行到了一个断点,
      
      \end{itemize}
    \end{block}

    \begin{block}{更多的问题}
      \begin{itemize}
      \item 编译器一般不会产生这样的指令,谁加入的?怎么加入的?原来的指令怎么办?
      \item 当程序执行到某个断点指令,产生trap时候,调试器怎么知道,到底是执行到了哪个断点,
      \item 调试器把断点指令加入的程序中,原来的程序还执行吗?怎么执行?
      \end{itemize}
    \end{block}

  \end{overprint}

}

\frame{
  \frametitle{断点}
  \lstinputlisting[language=C,caption={int3.c},captionpos={b},
  firstline=3,
  label={listing:breakpoint:int3}]
  {../../book/breakpoint/int3.c}

  \lstinputlisting[caption={output of simple-break},captionpos={b},
  label={listing:breakpoint:simple-break-output}]
  {../../book/breakpoint/simple-break-output}

  \begin{itemize}
    \item 展示一个20行的程序,处理断点。如Listing 2-1,2-2,2-3所示
    \item 调式器在处理断点的一些内部操作
  \end{itemize}

  \begin{overprint}

    \begin{block}{理解结果}
      \begin{itemize}
      \item 为什么收到SIGTRAP两次?
      \item 0x40000810 和 0x8048363 分别是什么地址?
      \end{itemize}
    \end{block}

  \end{overprint}

}

\frame{
  \frametitle{断点}

  \begin{overprint}

    \begin{block}{0x40000810 和 0x8048363 分别是什么地址?}
      \begin{itemize}
      \item abc
      \item 0
      \end{itemize}
      \lstinputlisting[caption={Disassembly the main() function},
      firstline=9,lastline=11,
      captionpos={b},label={listing:breakpoint:dis-int3}]
      {../../book/breakpoint/dis-int3}
    \end{block}

    \begin{block}{为什么收到SIGTRAP两次?}
      \begin{itemize}
      \item abc
      \item 0
      \end{itemize}
    \end{block}

  \end{overprint}
}

\frame{
  \frametitle{断点操作}

\begin{figure}
  \includegraphics{../../book/breakpoint/break.1}\hfill
  \includegraphics{../../book/breakpoint/break.2}\hfill
  \includegraphics{../../book/breakpoint/break.3}\hfill

%  \includegraphics{../../book/breakpoint/break.4}\hfill
%  \includegraphics{../../book/breakpoint/break.5}\hfill
%  \includegraphics{../../book/breakpoint/break.6}
  \caption{调试器设置断点}
  \label{fig:break}
\end{figure}
}

\frame{
  \frametitle{断点操作}

\begin{figure}
  \includegraphics{../../book/breakpoint/break.4}\hfill
  \includegraphics{../../book/breakpoint/break.5}\hfill
  \includegraphics{../../book/breakpoint/break.6}\hfill
  \includegraphics{../../book/breakpoint/break.2}
  \caption{调试器设置断点}
  \label{fig:break}
\end{figure}
}

\subsection{共享对象}
\frame{
  \frametitle{共享对象}
  \begin{itemize}
    \item调式器怎样处理共享对象。
    \item共享对象的加载
  \end{itemize}
}

\subsection{线程}
\frame{
  \frametitle{线程}
  \begin{itemize}
    \item线程的实现,以及对调式的支持(nptl-db)
    \item调试器对线程的处理
  \end{itemize}
}


\end{CJK}
\end{document}
